HOW TO WORKS JavaScript
JavaScript is a high-level programming language primarily used for web development. When a user accesses a web page, the browser retrieves the JavaScript source code. The process of executing JavaScript can be summarized in the following steps:    

Parsing: The browser first parses the JavaScript code, converting it into a format known as the Abstract Syntax Tree (AST). The AST represents the structure of the code in a way that the engine can understand.    

Compilation: Although JavaScript is traditionally an interpreted language, modern JavaScript engines, like Google's V8, use Just-In-Time (JIT) compilation. This means that the code is compiled into machine code at runtime, rather than being fully interpreted line by line. This allows for optimizations that improve performance.    

Execution: The JavaScript engine executes the compiled machine code. During this process, the engine may optimize the code further, especially for functions that are called frequently.    

Output: Finally, the results of the executed code are rendered in the browser, allowing users to see the output of the JavaScript code.    

In summary, JavaScript is a high-level language that is parsed into an AST, compiled into machine code using JIT compilation, and executed by the JavaScript engine, which optimizes the code for better performance before displaying the output to the user.